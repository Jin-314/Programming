\documentclass[dvipdfmx]{beamer}
%テーマの設定
\usetheme[numbering=fullbar]{focus}

%パッケージの設定
\usepackage{graphicx}
\usepackage{pxjahyper}
\usepackage{hyperref}
\usepackage{tikz}
\usetikzlibrary{arrows.meta,positioning}
\usepackage{listings, jlisting}
\usepackage{listings-arduino, listings-html5, listings-javascript}
\usepackage{xcolor}

%色設定
\definecolor{identifier}{RGB}{225, 123, 52}       %#e17b34
\definecolor{comment}{RGB}{105, 130, 27}          %#69821b
\definecolor{keyword}{RGB}{170, 76, 143}          %#aa4c8f
\definecolor{string}{RGB}{39, 146, 195}           %#2792c3

%ソースコード設定
\lstset{
  basicstyle={\ttfamily\footnotesize},                         % 基礎の文字のフォント設定
  identifierstyle={\color{identifier}},           % 変数名などのフォント設定
  commentstyle   ={\color{comment}},              % コメントのフォント設定
  keywordstyle   =[1]{\color{keyword}},           % 予約語のフォント設定
  stringstyle    ={\ttfamily\color{string}},      % 文字列のフォント設定
  tabsize=2
}

%フォント設定
\renewcommand{\kanjifamilydefault}{\gtdefault}

%メタ設定
\title{多言語プログラミング}
\subtitle{プログラミングにおける骨格}
\author[電研]{電気技術研究会}
\institute[NITNC]{NIT, Nara Collage}
\date[\today]{\today}
\titlegraphic{
  \begin{tikzpicture}[overlay, remember picture]
    \node[at=(current page.north east), anchor=north east]{
      \includegraphics[height=0.15\textheight]{pic/Logo(2).png}
      \hspace{0.05\textwidth}
    };
  \end{tikzpicture}
}

\begin{document}

\begin{frame}[plain]
  \titlepage
\end{frame}

\section*{概要}
  \begin{frame}[plain,noframenumbering]
    \sectionpage
  \end{frame}
  \begin{frame}{講義目的と進め方}
    \begin{large}講義目的\end{large}
    \begin{itemize}
      \setlength{\itemsep}{5mm}
      \item \structure{多言語}にわたるプログラミングを\structure{独学する}術を身に付ける
      \item 基本的な\structure{用語}や\structure{技術}を学び、\structure{活かせる力}を身に付ける
      \item マイコンやウェブなど、\structure{広い範囲に適応できる}人材育成
    \end{itemize}
    \vfill
    \begin{large}進め方\end{large}
    \begin{itemize}
      \setlength{\itemsep}{5mm}
      \item 重要な\structure{章}ごとにまとめて講義
      \item 1日(\structure{2~3時間})\structure{1~2}章分を進めたい
      \item 各章ごとに\structure{演習}がある
    \end{itemize}
  \end{frame}

  \begin{frame}{カリキュラム}
    \begin{tikzpicture}
      [every node/.style={rounded corners,inner sep=5pt, minimum width=100pt,minimum height=35pt},
      pre-subj/.style={fill=blue!20},post-subj/.style={fill=black!20},overlay, remember picture]
      \node[pre-subj] (a) at (0.15\textwidth,0.375\textheight){\hyperlink{programming}{プログラミング}};
      \node[pre-subj,below=10pt of a] (b) {\hyperlink{hensu}{変数とデータ型}};
      \node[pre-subj,below=10pt of b] (c) {\hyperlink{enzan}{演算と演算子}};
      \node[pre-subj,below=10pt of c] (d) {\hyperlink{seigyo}{制御構文}};
      \node[pre-subj,below=10pt of d] (e) {\hyperlink{hairetsu}{配列}};
      \node[pre-subj,right=10pt of e] (f) {\hyperlink{object}{オブジェクト}};
      \node[pre-subj,above=10pt of f] (g) {\hyperlink{kansu}{関数}};
      \node[pre-subj,above=10pt of g] (h) {\hyperlink{pointa}{ポインタ}};
      \node[pre-subj,above=10pt of h] (i) {\hyperlink{objectbun}{オブジェクト構文}};
      \node[pre-subj,above=10pt of i] (j) {\hyperlink{computer}{コンピュータ基礎}};
      \node[pre-subj,right=10pt of j] (k) {\hyperlink{micom}{マイコン}};
      \node[pre-subj,below=10pt of k] (l) {\hyperlink{iot}{IoT}};
      \node[pre-subj,below=10pt of l] (m) {\hyperlink{analys}{プログラム解析}};
      \node[pre-subj,below=10pt of m] (n) {\hyperlink{readability}{コードの可読性}};
      \node[pre-subj,below=10pt of n] (o) {\hyperlink{learn}{言語の学習法}};
      \foreach \x / \y in {a/b,b/c,c/d,d/e,e/f,f/g,g/h,h/i,i/j,j/k,k/l,l/m,m/n,n/o}{
        \draw[->,>={Stealth[round]},line width=.5mm] (\x) -- (\y);
      }
    \end{tikzpicture}
  \end{frame}

\section*{プログラミングとは}
  \begin{frame}[plain,noframenumbering]
    \label{programming}
    \sectionpage
  \end{frame}
  \begin{frame}{進度}
    \begin{tikzpicture}
      [every node/.style={rounded corners,inner sep=5pt, minimum width=100pt,minimum height=35pt},
      pre-subj/.style={fill=blue!20},now-subj/.style={white,fill=black!50!green},post-subj/.style={fill=black!20},overlay, remember picture]
      \node[now-subj] (a) at (0.15\textwidth,0.375\textheight){\hyperlink{programming}{プログラミング}};
      \node[pre-subj,below=10pt of a] (b) {\hyperlink{hensu}{変数とデータ型}};
      \node[pre-subj,below=10pt of b] (c) {\hyperlink{enzan}{演算と演算子}};
      \node[pre-subj,below=10pt of c] (d) {\hyperlink{seigyo}{制御構文}};
      \node[pre-subj,below=10pt of d] (e) {\hyperlink{hairetsu}{配列}};
      \node[pre-subj,right=10pt of e] (f) {\hyperlink{object}{オブジェクト}};
      \node[pre-subj,above=10pt of f] (g) {\hyperlink{kansu}{関数}};
      \node[pre-subj,above=10pt of g] (h) {\hyperlink{pointa}{ポインタ}};
      \node[pre-subj,above=10pt of h] (i) {\hyperlink{objectbun}{オブジェクト構文}};
      \node[pre-subj,above=10pt of i] (j) {\hyperlink{computer}{コンピュータ基礎}};
      \node[pre-subj,right=10pt of j] (k) {\hyperlink{micom}{マイコン}};
      \node[pre-subj,below=10pt of k] (l) {\hyperlink{iot}{IoT}};
      \node[pre-subj,below=10pt of l] (m) {\hyperlink{analys}{プログラム解析}};
      \node[pre-subj,below=10pt of m] (n) {\hyperlink{readability}{コードの可読性}};
      \node[pre-subj,below=10pt of n] (o) {\hyperlink{learn}{言語の学習法}};
      \foreach \x / \y in {a/b,b/c,c/d,d/e,e/f,f/g,g/h,h/i,i/j,j/k,k/l,l/m,m/n,n/o}{
        \draw[->,>={Stealth[round]},line width=.5mm] (\x) -- (\y);
      }
    \end{tikzpicture}
  \end{frame}

  \begin{frame}{プログラミングについて}
    \begin{large}プログラミングどんなもの?\end{large}
    \begin{itemize}
      \setlength{\itemsep}{3mm}
      \item プログラミングには様々な種類が存在\\ウェブ、マイコン、ゲーム...etc
      \item 大まかには2種類\\\structure{オブジェクト指向}と\structure{構造化プログラミング}(後述)
      \item \structure{骨格}は基本的に共通\\関数や変数など、基本的な\structure{骨格}がある
    \end{itemize}
    \vfill
    \begin{large}プログラミング言語\end{large}
    \begin{itemize}
      \setlength{\itemsep}{3mm}
      \item \structure{言語}は多数ある\par C, C++, C\#, JavaScript, VisualBasic, Python, \LaTeX
      \item 本教材では主に\structure{JavsScript}とArduinoによる\structure{C++}を使用
    \end{itemize}
  \end{frame}

  \begin{frame}[fragile]{基本的な骨格とは}
    \begin{columns}
      \begin{column}{0.6\textwidth}
        \begin{itemize}
          \item 冒頭で\structure{ライブラリ}の宣言
          \item \structure{メインループ}で処理\\UI等の場合では\structure{イベント}を処理
          \item \structure{サブルーチン(関数)}にて\\繰り返し処理
        \end{itemize}
        \centering
        \begin{minipage}{0.8\textwidth}
          \begin{exampleblock}{C言語の例}
            \begin{lstlisting}[language=C]
#include <stdio.h>

int main(void){
  printf('Hello World');
  return 0;
}    
            \end{lstlisting}
          \end{exampleblock}
        \end{minipage}
      \end{column}
      \begin{column}{0.4\textwidth}
        \begin{exampleblock}{Arduinoの例}
          \begin{lstlisting}[language=Arduino]
void setup(){
  pinMode(2, OUTPUT);
}
void loop(){
  digitalWrite(2, HIGH);
  delay(200);
  digitalWrite(2,LOW);
  delay(200);
}

          \end{lstlisting}
        \end{exampleblock}
      \end{column}
    \end{columns}
  \end{frame}

  \begin{frame}[fragile]{演習1:問題}{プログラミングとは}
    \begin{columns}
      \begin{column}{0.55\textwidth}
        \begin{large}JavaScript動作確認\end{large}
        \begin{enumerate}
          \setlength{\itemsep}{3mm}
          \item \structure{VSCode}の作業フォルダを作成
          \item {\color{string}'Hello World'}を表示させる
        \end{enumerate}
        \vspace{5mm}
        \begin{large}HTMLファイルへの埋め込み\end{large}
        \begin{itemize}
          \setlength{\itemsep}{3mm}
          \item \structure{html}ファイルの\structure{<script>}タグ内
          \item ディスプレイさせる関数\\
          \lstinline[language=javascript]|document.writeln('|{\color{string}文字列 '}\lstinline[language=javascript]|)|
          \item 文末には\structure{;}(\structure{セミコロン})をつける
        \end{itemize}
      \end{column}
      \begin{column}{0.45\textwidth}
        \begin{exampleblock}{htmlテンプレート}
          \begin{lstlisting}[language=html5,basicstyle=\tiny]
<html>
<head>
<meta
  http-equiv="Content-Type"
  content="text/html; charset=UTF-8">
<title>Hello,World!</title>
</head>
<body>
<pre>
<script type="text/javascript">
  // ここにスクリプトを記述
</script>
<noscript>
  JavaScriptが利用できません。
</noscript>
</pre>
</body>
</html>

          \end{lstlisting}
        \end{exampleblock}
      \end{column}
    \end{columns}
  \end{frame}

  \begin{frame}{演習1:回答}{プログラミングとは}
    \begin{exampleblock}{演習1}
      \lstinputlisting[language=javascript]{Ensyu/Ensyu1/Ensyu1.js}
    \end{exampleblock}
  \end{frame}

\section*{変数とデータ型}
  \begin{frame}[plain,noframenumbering]
    \label{hensu}
    \sectionpage
  \end{frame}

  \begin{frame}{進度}
    \begin{tikzpicture}
      [every node/.style={rounded corners,inner sep=5pt, minimum width=100pt,minimum height=35pt},
      pre-subj/.style={fill=blue!20},now-subj/.style={white,fill=black!50!green},post-subj/.style={fill=black!20},overlay, remember picture]
      \node[post-subj] (a) at (0.15\textwidth,0.375\textheight){\hyperlink{programming}{プログラミング}};
      \node[now-subj,below=10pt of a] (b) {\hyperlink{hensu}{変数とデータ型}};
      \node[pre-subj,below=10pt of b] (c) {\hyperlink{enzan}{演算と演算子}};
      \node[pre-subj,below=10pt of c] (d) {\hyperlink{seigyo}{制御構文}};
      \node[pre-subj,below=10pt of d] (e) {\hyperlink{hairetsu}{配列}};
      \node[pre-subj,right=10pt of e] (f) {\hyperlink{object}{オブジェクト}};
      \node[pre-subj,above=10pt of f] (g) {\hyperlink{kansu}{関数}};
      \node[pre-subj,above=10pt of g] (h) {\hyperlink{pointa}{ポインタ}};
      \node[pre-subj,above=10pt of h] (i) {\hyperlink{objectbun}{オブジェクト構文}};
      \node[pre-subj,above=10pt of i] (j) {\hyperlink{computer}{コンピュータ基礎}};
      \node[pre-subj,right=10pt of j] (k) {\hyperlink{micom}{マイコン}};
      \node[pre-subj,below=10pt of k] (l) {\hyperlink{iot}{IoT}};
      \node[pre-subj,below=10pt of l] (m) {\hyperlink{analys}{プログラム解析}};
      \node[pre-subj,below=10pt of m] (n) {\hyperlink{readability}{コードの可読性}};
      \node[pre-subj,below=10pt of n] (o) {\hyperlink{learn}{言語の学習法}};
      \foreach \x / \y in {a/b,b/c,c/d,d/e,e/f,f/g,g/h,h/i,i/j,j/k,k/l,l/m,m/n,n/o}{
        \draw[->,>={Stealth[round]},line width=.5mm] (\x) -- (\y);
      }
    \end{tikzpicture}
  \end{frame}
  
\section*{演算と演算子}
  \begin{frame}[plain,noframenumbering]
    \label{enzan}
    \sectionpage
  \end{frame}
  
  \begin{frame}{進度}
    \begin{tikzpicture}
      [every node/.style={rounded corners,inner sep=5pt, minimum width=100pt,minimum height=35pt},
      pre-subj/.style={fill=blue!20},now-subj/.style={white,fill=black!50!green},post-subj/.style={fill=black!20},overlay, remember picture]
      \node[post-subj] (a) at (0.15\textwidth,0.375\textheight){\hyperlink{programming}{プログラミング}};
      \node[post-subj,below=10pt of a] (b) {\hyperlink{hensu}{変数とデータ型}};
      \node[now-subj,below=10pt of b] (c) {\hyperlink{enzan}{演算と演算子}};
      \node[pre-subj,below=10pt of c] (d) {\hyperlink{seigyo}{制御構文}};
      \node[pre-subj,below=10pt of d] (e) {\hyperlink{hairetsu}{配列}};
      \node[pre-subj,right=10pt of e] (f) {\hyperlink{object}{オブジェクト}};
      \node[pre-subj,above=10pt of f] (g) {\hyperlink{kansu}{関数}};
      \node[pre-subj,above=10pt of g] (h) {\hyperlink{pointa}{ポインタ}};
      \node[pre-subj,above=10pt of h] (i) {\hyperlink{objectbun}{オブジェクト構文}};
      \node[pre-subj,above=10pt of i] (j) {\hyperlink{computer}{コンピュータ基礎}};
      \node[pre-subj,right=10pt of j] (k) {\hyperlink{micom}{マイコン}};
      \node[pre-subj,below=10pt of k] (l) {\hyperlink{iot}{IoT}};
      \node[pre-subj,below=10pt of l] (m) {\hyperlink{analys}{プログラム解析}};
      \node[pre-subj,below=10pt of m] (n) {\hyperlink{readability}{コードの可読性}};
      \node[pre-subj,below=10pt of n] (o) {\hyperlink{learn}{言語の学習法}};
      \foreach \x / \y in {a/b,b/c,c/d,d/e,e/f,f/g,g/h,h/i,i/j,j/k,k/l,l/m,m/n,n/o}{
        \draw[->,>={Stealth[round]},line width=.5mm] (\x) -- (\y);
      }
    \end{tikzpicture}
  \end{frame}

\section*{制御構文}
  \begin{frame}[plain,noframenumbering]
    \label{seigyo}
    \sectionpage
  \end{frame}
  
  \begin{frame}{進度}
    \begin{tikzpicture}
      [every node/.style={rounded corners,inner sep=5pt, minimum width=100pt,minimum height=35pt},
      pre-subj/.style={fill=blue!20},now-subj/.style={white,fill=black!50!green},post-subj/.style={fill=black!20},overlay, remember picture]
      \node[post-subj] (a) at (0.15\textwidth,0.375\textheight){\hyperlink{programming}{プログラミング}};
      \node[post-subj,below=10pt of a] (b) {\hyperlink{hensu}{変数とデータ型}};
      \node[post-subj,below=10pt of b] (c) {\hyperlink{enzan}{演算と演算子}};
      \node[now-subj,below=10pt of c] (d) {\hyperlink{seigyo}{制御構文}};
      \node[pre-subj,below=10pt of d] (e) {\hyperlink{hairetsu}{配列}};
      \node[pre-subj,right=10pt of e] (f) {\hyperlink{object}{オブジェクト}};
      \node[pre-subj,above=10pt of f] (g) {\hyperlink{kansu}{関数}};
      \node[pre-subj,above=10pt of g] (h) {\hyperlink{pointa}{ポインタ}};
      \node[pre-subj,above=10pt of h] (i) {\hyperlink{objectbun}{オブジェクト構文}};
      \node[pre-subj,above=10pt of i] (j) {\hyperlink{computer}{コンピュータ基礎}};
      \node[pre-subj,right=10pt of j] (k) {\hyperlink{micom}{マイコン}};
      \node[pre-subj,below=10pt of k] (l) {\hyperlink{iot}{IoT}};
      \node[pre-subj,below=10pt of l] (m) {\hyperlink{analys}{プログラム解析}};
      \node[pre-subj,below=10pt of m] (n) {\hyperlink{readability}{コードの可読性}};
      \node[pre-subj,below=10pt of n] (o) {\hyperlink{learn}{言語の学習法}};
      \foreach \x / \y in {a/b,b/c,c/d,d/e,e/f,f/g,g/h,h/i,i/j,j/k,k/l,l/m,m/n,n/o}{
        \draw[->,>={Stealth[round]},line width=.5mm] (\x) -- (\y);
      }
    \end{tikzpicture}
  \end{frame}

\section*{配列}
  \begin{frame}[plain,noframenumbering]
    \label{hairetsu}
    \sectionpage
  \end{frame}
  
  \begin{frame}{進度}
    \begin{tikzpicture}
      [every node/.style={rounded corners,inner sep=5pt, minimum width=100pt,minimum height=35pt},
      pre-subj/.style={fill=blue!20},now-subj/.style={white,fill=black!50!green},post-subj/.style={fill=black!20},overlay, remember picture]
      \node[post-subj] (a) at (0.15\textwidth,0.375\textheight){\hyperlink{programming}{プログラミング}};
      \node[post-subj,below=10pt of a] (b) {\hyperlink{hensu}{変数とデータ型}};
      \node[post-subj,below=10pt of b] (c) {\hyperlink{enzan}{演算と演算子}};
      \node[post-subj,below=10pt of c] (d) {\hyperlink{seigyo}{制御構文}};
      \node[now-subj,below=10pt of d] (e) {\hyperlink{hairetsu}{配列}};
      \node[pre-subj,right=10pt of e] (f) {\hyperlink{object}{オブジェクト}};
      \node[pre-subj,above=10pt of f] (g) {\hyperlink{kansu}{関数}};
      \node[pre-subj,above=10pt of g] (h) {\hyperlink{pointa}{ポインタ}};
      \node[pre-subj,above=10pt of h] (i) {\hyperlink{objectbun}{オブジェクト構文}};
      \node[pre-subj,above=10pt of i] (j) {\hyperlink{computer}{コンピュータ基礎}};
      \node[pre-subj,right=10pt of j] (k) {\hyperlink{micom}{マイコン}};
      \node[pre-subj,below=10pt of k] (l) {\hyperlink{iot}{IoT}};
      \node[pre-subj,below=10pt of l] (m) {\hyperlink{analys}{プログラム解析}};
      \node[pre-subj,below=10pt of m] (n) {\hyperlink{readability}{コードの可読性}};
      \node[pre-subj,below=10pt of n] (o) {\hyperlink{learn}{言語の学習法}};
      \foreach \x / \y in {a/b,b/c,c/d,d/e,e/f,f/g,g/h,h/i,i/j,j/k,k/l,l/m,m/n,n/o}{
        \draw[->,>={Stealth[round]},line width=.5mm] (\x) -- (\y);
      }
    \end{tikzpicture}
  \end{frame}

\section*{オブジェクト}
  \begin{frame}[plain,noframenumbering]
    \label{object}
    \sectionpage
  \end{frame}
  
  \begin{frame}{進度}
    \begin{tikzpicture}
      [every node/.style={rounded corners,inner sep=5pt, minimum width=100pt,minimum height=35pt},
      pre-subj/.style={fill=blue!20},now-subj/.style={white,fill=black!50!green},post-subj/.style={fill=black!20},overlay, remember picture]
      \node[post-subj] (a) at (0.15\textwidth,0.375\textheight){\hyperlink{programming}{プログラミング}};
      \node[post-subj,below=10pt of a] (b) {\hyperlink{hensu}{変数とデータ型}};
      \node[post-subj,below=10pt of b] (c) {\hyperlink{enzan}{演算と演算子}};
      \node[post-subj,below=10pt of c] (d) {\hyperlink{seigyo}{制御構文}};
      \node[post-subj,below=10pt of d] (e) {\hyperlink{hairetsu}{配列}};
      \node[now-subj,right=10pt of e] (f) {\hyperlink{object}{オブジェクト}};
      \node[pre-subj,above=10pt of f] (g) {\hyperlink{kansu}{関数}};
      \node[pre-subj,above=10pt of g] (h) {\hyperlink{pointa}{ポインタ}};
      \node[pre-subj,above=10pt of h] (i) {\hyperlink{objectbun}{オブジェクト構文}};
      \node[pre-subj,above=10pt of i] (j) {\hyperlink{computer}{コンピュータ基礎}};
      \node[pre-subj,right=10pt of j] (k) {\hyperlink{micom}{マイコン}};
      \node[pre-subj,below=10pt of k] (l) {\hyperlink{iot}{IoT}};
      \node[pre-subj,below=10pt of l] (m) {\hyperlink{analys}{プログラム解析}};
      \node[pre-subj,below=10pt of m] (n) {\hyperlink{readability}{コードの可読性}};
      \node[pre-subj,below=10pt of n] (o) {\hyperlink{learn}{言語の学習法}};
      \foreach \x / \y in {a/b,b/c,c/d,d/e,e/f,f/g,g/h,h/i,i/j,j/k,k/l,l/m,m/n,n/o}{
        \draw[->,>={Stealth[round]},line width=.5mm] (\x) -- (\y);
      }
    \end{tikzpicture}
  \end{frame}

\section*{関数}
  \begin{frame}[plain,noframenumbering]
    \label{kansu}
    \sectionpage
  \end{frame}
  
  \begin{frame}{進度}
    \begin{tikzpicture}
      [every node/.style={rounded corners,inner sep=5pt, minimum width=100pt,minimum height=35pt},
      pre-subj/.style={fill=blue!20},now-subj/.style={white,fill=black!50!green},post-subj/.style={fill=black!20},overlay, remember picture]
      \node[post-subj] (a) at (0.15\textwidth,0.375\textheight){\hyperlink{programming}{プログラミング}};
      \node[post-subj,below=10pt of a] (b) {\hyperlink{hensu}{変数とデータ型}};
      \node[post-subj,below=10pt of b] (c) {\hyperlink{enzan}{演算と演算子}};
      \node[post-subj,below=10pt of c] (d) {\hyperlink{seigyo}{制御構文}};
      \node[post-subj,below=10pt of d] (e) {\hyperlink{hairetsu}{配列}};
      \node[post-subj,right=10pt of e] (f) {\hyperlink{object}{オブジェクト}};
      \node[now-subj,above=10pt of f] (g) {\hyperlink{kansu}{関数}};
      \node[pre-subj,above=10pt of g] (h) {\hyperlink{pointa}{ポインタ}};
      \node[pre-subj,above=10pt of h] (i) {\hyperlink{objectbun}{オブジェクト構文}};
      \node[pre-subj,above=10pt of i] (j) {\hyperlink{computer}{コンピュータ基礎}};
      \node[pre-subj,right=10pt of j] (k) {\hyperlink{micom}{マイコン}};
      \node[pre-subj,below=10pt of k] (l) {\hyperlink{iot}{IoT}};
      \node[pre-subj,below=10pt of l] (m) {\hyperlink{analys}{プログラム解析}};
      \node[pre-subj,below=10pt of m] (n) {\hyperlink{readability}{コードの可読性}};
      \node[pre-subj,below=10pt of n] (o) {\hyperlink{learn}{言語の学習法}};
      \foreach \x / \y in {a/b,b/c,c/d,d/e,e/f,f/g,g/h,h/i,i/j,j/k,k/l,l/m,m/n,n/o}{
        \draw[->,>={Stealth[round]},line width=.5mm] (\x) -- (\y);
      }
    \end{tikzpicture}
  \end{frame}

\section*{ポインタ}
  \begin{frame}[plain,noframenumbering]
    \label{pointa}
    \sectionpage
  \end{frame}

  \begin{frame}{進度}
    \begin{tikzpicture}
      [every node/.style={rounded corners,inner sep=5pt, minimum width=100pt,minimum height=35pt},
      pre-subj/.style={fill=blue!20},now-subj/.style={white,fill=black!50!green},post-subj/.style={fill=black!20},overlay, remember picture]
      \node[post-subj] (a) at (0.15\textwidth,0.375\textheight){\hyperlink{programming}{プログラミング}};
      \node[post-subj,below=10pt of a] (b) {\hyperlink{hensu}{変数とデータ型}};
      \node[post-subj,below=10pt of b] (c) {\hyperlink{enzan}{演算と演算子}};
      \node[post-subj,below=10pt of c] (d) {\hyperlink{seigyo}{制御構文}};
      \node[post-subj,below=10pt of d] (e) {\hyperlink{hairetsu}{配列}};
      \node[post-subj,right=10pt of e] (f) {\hyperlink{object}{オブジェクト}};
      \node[post-subj,above=10pt of f] (g) {\hyperlink{kansu}{関数}};
      \node[now-subj,above=10pt of g] (h) {\hyperlink{pointa}{ポインタ}};
      \node[pre-subj,above=10pt of h] (i) {\hyperlink{objectbun}{オブジェクト構文}};
      \node[pre-subj,above=10pt of i] (j) {\hyperlink{computer}{コンピュータ基礎}};
      \node[pre-subj,right=10pt of j] (k) {\hyperlink{micom}{マイコン}};
      \node[pre-subj,below=10pt of k] (l) {\hyperlink{iot}{IoT}};
      \node[pre-subj,below=10pt of l] (m) {\hyperlink{analys}{プログラム解析}};
      \node[pre-subj,below=10pt of m] (n) {\hyperlink{readability}{コードの可読性}};
      \node[pre-subj,below=10pt of n] (o) {\hyperlink{learn}{言語の学習法}};
      \foreach \x / \y in {a/b,b/c,c/d,d/e,e/f,f/g,g/h,h/i,i/j,j/k,k/l,l/m,m/n,n/o}{
        \draw[->,>={Stealth[round]},line width=.5mm] (\x) -- (\y);
      }
    \end{tikzpicture}
  \end{frame}
  
\section*{オブジェクト指向構文}
  \begin{frame}[plain,noframenumbering]
    \label{objectbun}
    \sectionpage
  \end{frame}

  \begin{frame}{進度}
    \begin{tikzpicture}
      [every node/.style={rounded corners,inner sep=5pt, minimum width=100pt,minimum height=35pt},
      pre-subj/.style={fill=blue!20},now-subj/.style={white,fill=black!50!green},post-subj/.style={fill=black!20},overlay, remember picture]
      \node[post-subj] (a) at (0.15\textwidth,0.375\textheight){\hyperlink{programming}{プログラミング}};
      \node[post-subj,below=10pt of a] (b) {\hyperlink{hensu}{変数とデータ型}};
      \node[post-subj,below=10pt of b] (c) {\hyperlink{enzan}{演算と演算子}};
      \node[post-subj,below=10pt of c] (d) {\hyperlink{seigyo}{制御構文}};
      \node[post-subj,below=10pt of d] (e) {\hyperlink{hairetsu}{配列}};
      \node[post-subj,right=10pt of e] (f) {\hyperlink{object}{オブジェクト}};
      \node[post-subj,above=10pt of f] (g) {\hyperlink{kansu}{関数}};
      \node[post-subj,above=10pt of g] (h) {\hyperlink{pointa}{ポインタ}};
      \node[now-subj,above=10pt of h] (i) {\hyperlink{objectbun}{オブジェクト構文}};
      \node[pre-subj,above=10pt of i] (j) {\hyperlink{computer}{コンピュータ基礎}};
      \node[pre-subj,right=10pt of j] (k) {\hyperlink{micom}{マイコン}};
      \node[pre-subj,below=10pt of k] (l) {\hyperlink{iot}{IoT}};
      \node[pre-subj,below=10pt of l] (m) {\hyperlink{analys}{プログラム解析}};
      \node[pre-subj,below=10pt of m] (n) {\hyperlink{readability}{コードの可読性}};
      \node[pre-subj,below=10pt of n] (o) {\hyperlink{learn}{言語の学習法}};
      \foreach \x / \y in {a/b,b/c,c/d,d/e,e/f,f/g,g/h,h/i,i/j,j/k,k/l,l/m,m/n,n/o}{
        \draw[->,>={Stealth[round]},line width=.5mm] (\x) -- (\y);
      }
    \end{tikzpicture}
  \end{frame}

\section*{コンピュータ基礎}
  \begin{frame}[plain,noframenumbering]
    \label{computer}
    \sectionpage
  \end{frame}

  \begin{frame}{進度}
    \begin{tikzpicture}
      [every node/.style={rounded corners,inner sep=5pt, minimum width=100pt,minimum height=35pt},
      pre-subj/.style={fill=blue!20},now-subj/.style={white,fill=black!50!green},post-subj/.style={fill=black!20},overlay, remember picture]
      \node[post-subj] (a) at (0.15\textwidth,0.375\textheight){\hyperlink{programming}{プログラミング}};
      \node[post-subj,below=10pt of a] (b) {\hyperlink{hensu}{変数とデータ型}};
      \node[post-subj,below=10pt of b] (c) {\hyperlink{enzan}{演算と演算子}};
      \node[post-subj,below=10pt of c] (d) {\hyperlink{seigyo}{制御構文}};
      \node[post-subj,below=10pt of d] (e) {\hyperlink{hairetsu}{配列}};
      \node[post-subj,right=10pt of e] (f) {\hyperlink{object}{オブジェクト}};
      \node[post-subj,above=10pt of f] (g) {\hyperlink{kansu}{関数}};
      \node[post-subj,above=10pt of g] (h) {\hyperlink{pointa}{ポインタ}};
      \node[post-subj,above=10pt of h] (i) {\hyperlink{objectbun}{オブジェクト構文}};
      \node[now-subj,above=10pt of i] (j) {\hyperlink{computer}{コンピュータ基礎}};
      \node[pre-subj,right=10pt of j] (k) {\hyperlink{micom}{マイコン}};
      \node[pre-subj,below=10pt of k] (l) {\hyperlink{iot}{IoT}};
      \node[pre-subj,below=10pt of l] (m) {\hyperlink{analys}{プログラム解析}};
      \node[pre-subj,below=10pt of m] (n) {\hyperlink{readability}{コードの可読性}};
      \node[pre-subj,below=10pt of n] (o) {\hyperlink{learn}{言語の学習法}};
      \foreach \x / \y in {a/b,b/c,c/d,d/e,e/f,f/g,g/h,h/i,i/j,j/k,k/l,l/m,m/n,n/o}{
        \draw[->,>={Stealth[round]},line width=.5mm] (\x) -- (\y);
      }
    \end{tikzpicture}
  \end{frame}

\section*{マイコン}
  \begin{frame}[plain,noframenumbering]
    \label{micom}
    \sectionpage
  \end{frame}
  
  \begin{frame}{進度}
    \begin{tikzpicture}
      [every node/.style={rounded corners,inner sep=5pt, minimum width=100pt,minimum height=35pt},
      pre-subj/.style={fill=blue!20},now-subj/.style={white,fill=black!50!green},post-subj/.style={fill=black!20},overlay, remember picture]
      \node[post-subj] (a) at (0.15\textwidth,0.375\textheight){\hyperlink{programming}{プログラミング}};
      \node[post-subj,below=10pt of a] (b) {\hyperlink{hensu}{変数とデータ型}};
      \node[post-subj,below=10pt of b] (c) {\hyperlink{enzan}{演算と演算子}};
      \node[post-subj,below=10pt of c] (d) {\hyperlink{seigyo}{制御構文}};
      \node[post-subj,below=10pt of d] (e) {\hyperlink{hairetsu}{配列}};
      \node[post-subj,right=10pt of e] (f) {\hyperlink{object}{オブジェクト}};
      \node[post-subj,above=10pt of f] (g) {\hyperlink{kansu}{関数}};
      \node[post-subj,above=10pt of g] (h) {\hyperlink{pointa}{ポインタ}};
      \node[post-subj,above=10pt of h] (i) {\hyperlink{objectbun}{オブジェクト構文}};
      \node[post-subj,above=10pt of i] (j) {\hyperlink{computer}{コンピュータ基礎}};
      \node[now-subj,right=10pt of j] (k) {\hyperlink{micom}{マイコン}};
      \node[pre-subj,below=10pt of k] (l) {\hyperlink{iot}{IoT}};
      \node[pre-subj,below=10pt of l] (m) {\hyperlink{analys}{プログラム解析}};
      \node[pre-subj,below=10pt of m] (n) {\hyperlink{readability}{コードの可読性}};
      \node[pre-subj,below=10pt of n] (o) {\hyperlink{learn}{言語の学習法}};
      \foreach \x / \y in {a/b,b/c,c/d,d/e,e/f,f/g,g/h,h/i,i/j,j/k,k/l,l/m,m/n,n/o}{
        \draw[->,>={Stealth[round]},line width=.5mm] (\x) -- (\y);
      }
    \end{tikzpicture}
  \end{frame}

\section*{IoT}
  \begin{frame}[plain,noframenumbering]
    \label{iot}
    \sectionpage
  \end{frame}

  \begin{frame}{進度}
    \begin{tikzpicture}
      [every node/.style={rounded corners,inner sep=5pt, minimum width=100pt,minimum height=35pt},
      pre-subj/.style={fill=blue!20},now-subj/.style={white,fill=black!50!green},post-subj/.style={fill=black!20},overlay, remember picture]
      \node[post-subj] (a) at (0.15\textwidth,0.375\textheight){\hyperlink{programming}{プログラミング}};
      \node[post-subj,below=10pt of a] (b) {\hyperlink{hensu}{変数とデータ型}};
      \node[post-subj,below=10pt of b] (c) {\hyperlink{enzan}{演算と演算子}};
      \node[post-subj,below=10pt of c] (d) {\hyperlink{seigyo}{制御構文}};
      \node[post-subj,below=10pt of d] (e) {\hyperlink{hairetsu}{配列}};
      \node[post-subj,right=10pt of e] (f) {\hyperlink{object}{オブジェクト}};
      \node[post-subj,above=10pt of f] (g) {\hyperlink{kansu}{関数}};
      \node[post-subj,above=10pt of g] (h) {\hyperlink{pointa}{ポインタ}};
      \node[post-subj,above=10pt of h] (i) {\hyperlink{objectbun}{オブジェクト構文}};
      \node[post-subj,above=10pt of i] (j) {\hyperlink{computer}{コンピュータ基礎}};
      \node[post-subj,right=10pt of j] (k) {\hyperlink{micom}{マイコン}};
      \node[now-subj,below=10pt of k] (l) {\hyperlink{iot}{IoT}};
      \node[pre-subj,below=10pt of l] (m) {\hyperlink{analys}{プログラム解析}};
      \node[pre-subj,below=10pt of m] (n) {\hyperlink{readability}{コードの可読性}};
      \node[pre-subj,below=10pt of n] (o) {\hyperlink{learn}{言語の学習法}};
      \foreach \x / \y in {a/b,b/c,c/d,d/e,e/f,f/g,g/h,h/i,i/j,j/k,k/l,l/m,m/n,n/o}{
        \draw[->,>={Stealth[round]},line width=.5mm] (\x) -- (\y);
      }
    \end{tikzpicture}
  \end{frame}

\section*{プログラム解析}
  \begin{frame}[plain,noframenumbering]
    \label{analys}
    \sectionpage
  \end{frame}

  \begin{frame}{進度}
    \begin{tikzpicture}
      [every node/.style={rounded corners,inner sep=5pt, minimum width=100pt,minimum height=35pt},
      pre-subj/.style={fill=blue!20},now-subj/.style={white,fill=black!50!green},post-subj/.style={fill=black!20},overlay, remember picture]
      \node[post-subj] (a) at (0.15\textwidth,0.375\textheight){\hyperlink{programming}{プログラミング}};
      \node[post-subj,below=10pt of a] (b) {\hyperlink{hensu}{変数とデータ型}};
      \node[post-subj,below=10pt of b] (c) {\hyperlink{enzan}{演算と演算子}};
      \node[post-subj,below=10pt of c] (d) {\hyperlink{seigyo}{制御構文}};
      \node[post-subj,below=10pt of d] (e) {\hyperlink{hairetsu}{配列}};
      \node[post-subj,right=10pt of e] (f) {\hyperlink{object}{オブジェクト}};
      \node[post-subj,above=10pt of f] (g) {\hyperlink{kansu}{関数}};
      \node[post-subj,above=10pt of g] (h) {\hyperlink{pointa}{ポインタ}};
      \node[post-subj,above=10pt of h] (i) {\hyperlink{objectbun}{オブジェクト構文}};
      \node[post-subj,above=10pt of i] (j) {\hyperlink{computer}{コンピュータ基礎}};
      \node[post-subj,right=10pt of j] (k) {\hyperlink{micom}{マイコン}};
      \node[post-subj,below=10pt of k] (l) {\hyperlink{iot}{IoT}};
      \node[now-subj,below=10pt of l] (m) {\hyperlink{analys}{プログラム解析}};
      \node[pre-subj,below=10pt of m] (n) {\hyperlink{readability}{コードの可読性}};
      \node[pre-subj,below=10pt of n] (o) {\hyperlink{learn}{言語の学習法}};
      \foreach \x / \y in {a/b,b/c,c/d,d/e,e/f,f/g,g/h,h/i,i/j,j/k,k/l,l/m,m/n,n/o}{
        \draw[->,>={Stealth[round]},line width=.5mm] (\x) -- (\y);
      }
    \end{tikzpicture}
  \end{frame}

\section*{コードの可読性}
  \begin{frame}[plain,noframenumbering]
    \label{readability}
    \sectionpage
  \end{frame}

  \begin{frame}{進度}
    \begin{tikzpicture}
      [every node/.style={rounded corners,inner sep=5pt, minimum width=100pt,minimum height=35pt},
      pre-subj/.style={fill=blue!20},now-subj/.style={white,fill=black!50!green},post-subj/.style={fill=black!20},overlay, remember picture]
      \node[post-subj] (a) at (0.15\textwidth,0.375\textheight){\hyperlink{programming}{プログラミング}};
      \node[post-subj,below=10pt of a] (b) {\hyperlink{hensu}{変数とデータ型}};
      \node[post-subj,below=10pt of b] (c) {\hyperlink{enzan}{演算と演算子}};
      \node[post-subj,below=10pt of c] (d) {\hyperlink{seigyo}{制御構文}};
      \node[post-subj,below=10pt of d] (e) {\hyperlink{hairetsu}{配列}};
      \node[post-subj,right=10pt of e] (f) {\hyperlink{object}{オブジェクト}};
      \node[post-subj,above=10pt of f] (g) {\hyperlink{kansu}{関数}};
      \node[post-subj,above=10pt of g] (h) {\hyperlink{pointa}{ポインタ}};
      \node[post-subj,above=10pt of h] (i) {\hyperlink{objectbun}{オブジェクト構文}};
      \node[post-subj,above=10pt of i] (j) {\hyperlink{computer}{コンピュータ基礎}};
      \node[post-subj,right=10pt of j] (k) {\hyperlink{micom}{マイコン}};
      \node[post-subj,below=10pt of k] (l) {\hyperlink{iot}{IoT}};
      \node[post-subj,below=10pt of l] (m) {\hyperlink{analys}{プログラム解析}};
      \node[now-subj,below=10pt of m] (n) {\hyperlink{readability}{コードの可読性}};
      \node[pre-subj,below=10pt of n] (o) {\hyperlink{learn}{言語の学習法}};
      \foreach \x / \y in {a/b,b/c,c/d,d/e,e/f,f/g,g/h,h/i,i/j,j/k,k/l,l/m,m/n,n/o}{
        \draw[->,>={Stealth[round]},line width=.5mm] (\x) -- (\y);
      }
    \end{tikzpicture}
  \end{frame}

\section*{言語の学習法}
  \begin{frame}[plain,noframenumbering]
    \label{learn}
    \sectionpage
  \end{frame}
  
  \begin{frame}{進度}
    \begin{tikzpicture}
      [every node/.style={rounded corners,inner sep=5pt, minimum width=100pt,minimum height=35pt},
      pre-subj/.style={fill=blue!20},now-subj/.style={white,fill=black!50!green},post-subj/.style={fill=black!20},overlay, remember picture]
      \node[post-subj] (a) at (0.15\textwidth,0.375\textheight){\hyperlink{programming}{プログラミング}};
      \node[post-subj,below=10pt of a] (b) {\hyperlink{hensu}{変数とデータ型}};
      \node[post-subj,below=10pt of b] (c) {\hyperlink{enzan}{演算と演算子}};
      \node[post-subj,below=10pt of c] (d) {\hyperlink{seigyo}{制御構文}};
      \node[post-subj,below=10pt of d] (e) {\hyperlink{hairetsu}{配列}};
      \node[post-subj,right=10pt of e] (f) {\hyperlink{object}{オブジェクト}};
      \node[post-subj,above=10pt of f] (g) {\hyperlink{kansu}{関数}};
      \node[post-subj,above=10pt of g] (h) {\hyperlink{pointa}{ポインタ}};
      \node[post-subj,above=10pt of h] (i) {\hyperlink{objectbun}{オブジェクト構文}};
      \node[post-subj,above=10pt of i] (j) {\hyperlink{computer}{コンピュータ基礎}};
      \node[post-subj,right=10pt of j] (k) {\hyperlink{micom}{マイコン}};
      \node[post-subj,below=10pt of k] (l) {\hyperlink{iot}{IoT}};
      \node[post-subj,below=10pt of l] (m) {\hyperlink{analys}{プログラム解析}};
      \node[post-subj,below=10pt of m] (n) {\hyperlink{readability}{コードの可読性}};
      \node[now-subj,below=10pt of n] (o) {\hyperlink{learn}{言語の学習法}};
      \foreach \x / \y in {a/b,b/c,c/d,d/e,e/f,f/g,g/h,h/i,i/j,j/k,k/l,l/m,m/n,n/o}{
        \draw[->,>={Stealth[round]},line width=.5mm] (\x) -- (\y);
      }
    \end{tikzpicture}
  \end{frame}

\section*{問題解決能力}
  \begin{frame}[plain,noframenumbering]
    \sectionpage
  \end{frame}

\end{document}