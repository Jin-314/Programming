\documentclass[dvipdfmx]{beamer}
%テーマの設定
\usetheme[numbering=fullbar]{focus}
%パッケージの設定
\usepackage{graphicx}
\usepackage{pxjahyper}
\usepackage{tikz}
\usetikzlibrary{arrows.meta,positioning}

%フォント設定
\renewcommand{\kanjifamilydefault}{\gtdefault}

%メタ設定
\title{多言語プログラミング}
\subtitle{プログラミングにおける骨格}
\author[電研]{電気技術研究会}
\institute[NITNC]{NIT, Nara Collage}
\date[\today]{\today}
\titlegraphic{
  \begin{tikzpicture}[overlay, remember picture]
    \node[at=(current page.north east), anchor=north east]{
      \includegraphics[height=0.15\textheight]{pic/Logo(2).png}
      \hspace{0.05\textwidth}
    };
  \end{tikzpicture}
}

\begin{document}

\begin{frame}[plain]
  \titlepage
\end{frame}


\section{概要}

  \begin{frame}{講義目的と進め方}
    講義目的
    \begin{itemize}
      \setlength{\itemsep}{3mm}
      \item \structure{多言語}にわたるプログラミングを\structure{独学する}術を身に付ける
      \item 基本的な\structure{用語}や\structure{技術}を学び、\structure{活かせる力}を身に付ける
      \item マイコンやウェブなど、\structure{広い範囲に適応できる}人材育成
    \end{itemize}
    \vfill
    進め方    
    \begin{itemize}
      \setlength{\itemsep}{3mm}
      \item 重要な\structure{章}ごとにまとめて講義
      \item 1日(\structure{2~3時間})\structure{1~2}章分を進めたい
      \item 各章ごとに\structure{演習}がある
    \end{itemize}
  \end{frame}

  \begin{frame}{カリキュラム}
    \begin{tikzpicture}
      [every node/.style={rounded corners,inner sep=5pt, minimum width=100pt,minimum height=35pt},
      pre-subj/.style={fill=blue!20},post-subj/.style={fill=black!20},overlay, remember picture]
      \node[pre-subj] (a) at (0.15\textwidth,0.375\textheight){プログラミング};
      \node[pre-subj,below=10pt of a] (b) {変数とデータ型};
      \node[pre-subj,below=10pt of b] (c) {演算と演算子};
      \node[pre-subj,below=10pt of c] (d) {制御構文};
      \node[pre-subj,below=10pt of d] (e) {配列};
      \node[pre-subj,right=10pt of e] (f) {オブジェクト};
      \node[pre-subj,above=10pt of f] (g) {関数};
      \node[pre-subj,above=10pt of g] (h) {ポインタ};
      \node[pre-subj,above=10pt of h,align=center] (i) {オブジェクト\\指向構文};
      \node[pre-subj,above=10pt of i] (j) {コンピュータ基礎};
      \node[pre-subj,right=10pt of j,align=center] (k) {マイコン\\プログラミング};
      \node[pre-subj,below=10pt of k] (l) {IoT};
      \node[pre-subj,below=10pt of l,align=center] (m) {プログラミング\\解析};
      \node[pre-subj,below=10pt of m] (n) {コードの可読性};
      \node[pre-subj,below=10pt of n] (o) {言語の学習法};
      \foreach \x / \y in {a/b,b/c,c/d,d/e,e/f,f/g,g/h,h/i,i/j,j/k,k/l,l/m,m/n,n/o}{
        \draw[->,>={Stealth[round]},line width=.5mm] (\x) -- (\y);
      }
    \end{tikzpicture}
  \end{frame}

\section{プログラミングとは}
  \begin{frame}{進度}
    \begin{tikzpicture}
      [every node/.style={rounded corners,inner sep=5pt, minimum width=100pt,minimum height=35pt},
      pre-subj/.style={fill=blue!20},now-subj/.style={white,fill=black!50!green},post-subj/.style={fill=black!20},overlay, remember picture]
      \node[now-subj] (a) at (0.15\textwidth,0.375\textheight){プログラミング};
      \node[pre-subj,below=10pt of a] (b) {変数とデータ型};
      \node[pre-subj,below=10pt of b] (c) {演算と演算子};
      \node[pre-subj,below=10pt of c] (d) {制御構文};
      \node[pre-subj,below=10pt of d] (e) {配列};
      \node[pre-subj,right=10pt of e] (f) {オブジェクト};
      \node[pre-subj,above=10pt of f] (g) {関数};
      \node[pre-subj,above=10pt of g] (h) {ポインタ};
      \node[pre-subj,above=10pt of h,align=center] (i) {オブジェクト\\指向構文};
      \node[pre-subj,above=10pt of i] (j) {コンピュータ基礎};
      \node[pre-subj,right=10pt of j,align=center] (k) {マイコン\\プログラミング};
      \node[pre-subj,below=10pt of k] (l) {IoT};
      \node[pre-subj,below=10pt of l,align=center] (m) {プログラミング\\解析};
      \node[pre-subj,below=10pt of m] (n) {コードの可読性};
      \node[pre-subj,below=10pt of n] (o) {言語の学習法};
      \foreach \x / \y in {a/b,b/c,c/d,d/e,e/f,f/g,g/h,h/i,i/j,j/k,k/l,l/m,m/n,n/o}{
        \draw[->,>={Stealth[round]},line width=.5mm] (\x) -- (\y);
      }
    \end{tikzpicture}
  \end{frame}

  \begin{frame}{プログラミングについて}
    プログラミングどんなもの?
    \begin{itemize}
      \setlength{\itemsep}{3mm}
      \item プログラミングには様々な種類が存在\\ウェブ、マイコン、ゲーム...etc
      \item 大まかには2種類\\\structure{オブジェクト指向}と\structure{構造化プログラミング}(後述)
      \item \structure{骨格}は基本的に共通\\関数や変数など、基本的な\structure{骨格}がある
    \end{itemize}
    \vfill
    プログラミング言語
    \begin{itemize}
      \setlength{\itemsep}{3mm}
      \item \structure{言語}は多数ある\par C, C++, C\#, JavaScript, VisualBasic, Python, \LaTeX
      \item 本教材では主に\structure{JavsScript}とArduinoによる\structure{C++}を使用
    \end{itemize}
  \end{frame}

  \begin{frame}{基本的な骨格とは}
    \begin{itemize}
      \item 冒頭で\structure{ライブラリ}の宣言
      \item \structure{メインループ}で処理、UI等の場合では\structure{イベント}を処理
      \item \structure{サブルーチン(関数)}にて繰り返し処理
    \end{itemize}
  \end{frame}
\end{document}